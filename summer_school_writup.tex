\documentclass{article}
%%%%%%%%%%%%%%%%%%%%%%%%%%%%%%%%%%%
\title{Direct Monte Carlo Simulation of 1D Thermal Radiative Transfer}
\author{AUTHORS}

%\institute{INSTITUTIONS
%}

%\email{what what, the emails}

% Optional disclaimer: remove this command to hide
%\disclaimer{Notice: this manuscript is a work of fiction. Any resemblance to
%actual articles, living or dead, is purely coincidental.}

%%%% packages and definitions (optional)
\usepackage{graphicx} % allows inclusion of graphics
\usepackage{booktabs} % nice rules (thick lines) for tables
\usepackage{microtype} % improves typography for PDF
\usepackage{comment}
\usepackage{float}
\usepackage{amsmath}
\usepackage{setspace}

\newcommand{\SN}{S$_N$}
\renewcommand{\vec}[1]{\bm{#1}} %vector is bold italic
\newcommand{\vd}{\bm{\cdot}} % slightly bold vector dot
\newcommand{\grad}{\vec{\nabla}} % gradient
\newcommand{\ud}{\mathop{}\!\mathrm{d}} % upright derivative symbol
\renewcommand{\eqref}[1]{(\ref{#1})}
\renewcommand*\thetable{\Roman{table}}
\usepackage{caption}

\captionsetup[table]{labelsep=period}
\captionsetup[figure]{labelsep=period}
\newcommand{\N}{\mathbb{N}}
\newcommand{\Z}{\mathbb{Z}}
\newcommand{\deriv}[2]{\frac{\mathrm{d} #1}{\mathrm{d} #2}}
\newcommand{\pderiv}[2]{\frac{\partial #1}{\partial #2}}
\newcommand{\bx}{\mathbf{X}}
\newcommand{\ba}{\mathbf{A}}
\newcommand{\by}{\mathbf{Y}}
\newcommand{\bj}{\mathbf{J}}
\newcommand{\bs}{\mathbf{s}}
\newcommand{\B}[1]{\ensuremath{\mathbf{#1}}}
\newcommand{\Dt}{\Delta t}
\renewcommand{\d}{\mathrm{d}}
\newcommand{\mom}[1]{\langle #1 \rangle}
\newcommand{\il}{{i-1/2}}
\newcommand{\ir}{{i+1/2}}
\newcommand{\keff}{\ensuremath{k_{\text{eff}}}}
\newcommand{\sig}[1]{\ensuremath{\Sigma_{#1}}}
\newcommand{\ra}{\ensuremath{\rightarrow}}
\newcommand{\dd}{\ensuremath{\mathrm{d}}}
\renewcommand{\O}{\ensuremath{\mathbf{\Omega}}}
\newcommand{\adj}{\ensuremath{{}^{\dagger}}}
\newcommand{\x}{\ensuremath{\mathbf{x}}}
\newcommand{\T}{\ensuremath{\text{T}}}
\newcommand{\iso}[2]{$^{#2}$#1}
\newcommand{\phibar}{\ensuremath{\overline{\phi}}}
\newcommand{\cur}[1]{\left\{ #1 \right\}}
\newcommand{\E}[1]{\ensuremath{\operatorname{E}_{#1}}}
\newcommand{\rface}{\ensuremath{r_{\text{face}}}}
\newcommand{\FOM}{\ensuremath{\text{FOM}}}
\newcommand{\ds}[0]{\displaystyle}
\newcommand{\invcm}[0]{cm$^{-1}$}
\renewcommand{\u}[1]{\ensuremath{\underline{#1}}}
\newcommand{\dep}{\ensuremath{\delta\epsilon^{(m)}}}
\renewcommand{\ss}{\ensuremath{\|s\|}}
\newcommand{\pos}{{\text{pos}}}
\newcommand{\phsp}{\ensuremath{\left(\mathbf{r},\mathbf{\Omega},\nu,t\right)}}
\newcommand{\Del}{\ensuremath{\nabla}}

% commands specific to this paper
\newcommand{\jl}{{j-1/2}}
\newcommand{\jr}{{j+1/2}}
\newcommand{\xl}{{x_{j-1/2}}}
\newcommand{\xr}{{x_{j+1/2}}}
\newcommand{\ncl}{\ensuremath{N_{\text{cells}}}}
\newcommand{\Rin}{\ensuremath{R_{\text{int}}}}
\newcommand{\Rem}{\ensuremath{R_{\text{em}}}}
\newcommand{\Rin}{\ensuremath{R_{\text{int}}}}
\newcommand{\Remg}{\ensuremath{R_{\text{em}gj}}}

\begin{document}
\doublespacing

\section{Introduction}

We will be modeling 1D, frequency-dependent thermal radiative transfer problems,
in the HEDP regime, by performing direct Monte Carlo sampling of
the underlying physical distributions.  This method treats the absorption and
emission of photons on an event-by-event basis to advance the simulation
continuously in time and space, with opacities (i.e., cross sections) that are typically
multi-group.  This differs from the Implicit Monte Carlo
method~\cite{wollaber_review,wollaber_thesis}, which uses Monte
Carlo to solve a linear transport equation resulting from the linearization of
the thermal radiative transfer equations.  The two key approximations in IMC are
the linearization of the equations (linearized about the solution at a previous
time step), and the semi-implicit time discretization (if the non-linearities
were converged, it would be implicit, but that is too expensive).  

In the direct approach, as each event happens, the underlying probabilities of
each event occurring change.  The probabilities change because of the nonlinear relation between opacities
$\sigma$, temperature $T$, and the Planckian emission source $B(\nu,T)$ (which
is proportional to $T^4$.  The change in probabilities makes
this a difficult algorithm to parallelize, and the sampling of the non-linear
probability distribution functions (PDFs) results in high variance relative to
a linear Monte Carlo solution for the same dimensionality.

The direct method was implemented previously by Booth~\cite{booth2011} for
problems without spatial dependence.  The
goal for the summer is to implement a one-spatial dimension version of the
algorithm.  Although it will likely be practically inefficient compared to IMC, this code
will provide a valuable verification tool for certain IMC approximations (e.g.,
Compton scattering and potentially spatial and temporal discretization errors).  

\section{Outline}

A rough outline of this summmer is as follows (these are not necessarily
sequentially dependent):
\begin{enumerate}

\item Read~\cite{booth2011} to understand the algorithm in 0D.  
\item Understand the implicit Monte Carlo method and more importantly the
thermal radiative transfer equations.
Great references for this are~\cite{wollaber_review,wollaber_thesis}.  You will
not be implementing this, but it is good to have an idea of how \emph{literally}
everyone else solves this problem (for deterministic solution methods, cntrl-h
Monte Carlo with S$_n$, and add "iterate to convergence" at least twice per
page).
\item Implement the 0D algorithm.  We were thinking python first, and then c++.
But because of time constraints this may change towards a single language.
\begin{itemize}
\item 1-group
\item Multi-group
\item Add delta-scattering for acceleration
\item Incorporate the ability to read real opacities using Alex's tool
\item \textbf{Code Practices} Because of time constraints, we are
not planning on doing rigorous unit-testing (it is also not of huge value to a
project of this size and life-time), but if you want to test each function we
won't stop you.  We will use version control and review your code as it comes
in.  It would be great if we can rerun the integrated
test problems in the future easily to verify the code still works.
\end{itemize}
\item Run enough problems to be convinced it is working. 
\item We will provide you with a detailed 1D algorithm.  Try to think ahead
towards 1D as you implement the 0D code.
\item Implement the 1D algorithm.
\item Verify it works by running Marshak wave problems and comparing to Jayenne
(LANL IMC code). 
\item Implement Compton scattering and compare to IMC with compton scattering.
\item Explore domain copy parallelism (i.e., run the same simulation multiple
times on different processors and average the results) for statistical
efficiency.  Unlike typical linear Monte Carlo simulations, for the same number of histories you
will not get the same sample variance in a domain-copy simulation as if a single
simulation were ran.
However, at sufficiently high history counts, histories may become independent enough that
the benefit is sufficient to justify this trivial approach.
\end{enumerate}

As we have never hosted students from the summer school before and don't know
exactly how much time you will have, if we don't make
it to this point, that is OK.  However, if you guys are rock stars, then the
following stretch goals should be interesting:

\begin{itemize}

\item Try non-analog diffusion-approximation acceleration.
\item Compare point-wise re-emission to a linear spatial dependence.  This will
be insanely memory intensive and increase variance greatly, but it will be useful to show that linear is
good enough in most cases for a smaller set of problems.  
\item We could incorporate exponential spatial discretizations for certain
regions, although everyone could do this and doesn't because it is annoying and
doesn't usually matter.
\item Incorporate GPU acceleration to particle movements and where-else possible
(Alex Long will lead this...I don't computers).
\item Add lines and continuous spectrum cross sections.
\item Perform variance reduction techniques for conditional sampling of
different lines or events of interest
\item Perform variance reduction techniques or approximations that allows for increased
parallelism of the algorithm.  This is a hard one...we haven't really thought of
any good ones yet...so if you guys have ideas...

\end{itemize}

\section{1D Delta-Scattering Algorithm}

The inclusion of delta scattering allows for a decrease in the number of opacity
look ups and particle movements required to simulate the problem in 1D. Assume
there are $N$ photons, numbered for $k=1,2,\ldots,N$, multigroup equations for
$g=1,\ldots,G$ groups, and a 1D spatial domain of cells with cell $j$ defined
for $\xl \leq x \leq \xr$.  The intensity in the $j$-th cell can be estimated, at some time
$t$, as:




  The 1D
algorithm, with delta scattering, is as follows

\begin{enumerate}
\item Initialize problem by sampling locations of $N$ photons according to
initial conditions.
\item Set the current simulation time to $t_{c}=t_0$.
\item Compute time for transport to non-reflective \emph{domain} boundaries 
for each photon (or only those photons that are in cells next to boundaries
probably) as $t_{lkg,k} = t_c + \|x_{bnd} - x_k\|/c$.  
\item Find minimum time $t_{\lkg}$ = \min\left\{t_{lkg,k\right\}.
\begin{itemize}
\item Keep a ordered list of the top few percent of photons that are nearest to having a
leakage event.  As the photon with $t_{lkg,\min}$ has an absorption/scattering event, we don't
want to recompute this.
\item Compute total opacity $\sigma_t(\nu,T)$ for each photon, and compute the
maximum opacity
\begin{equation}
   \sigma_{\max} = \max_k\left\{\sigma_{t,k}(\nu,T)\right\}.
\end{equation}
Assuming that opacity is strictly inversely proportional to temperature and only
heating occurs, only emitted or scattered photons entering a new group can
increase the value of $\sigma_\max$.  A more general approach is probably needed
when this assumption is not true.
\item Compute Planckian emission distributions, for each cell.
\item Compute event production rates for each type of event
\begin{itemize}
  \item Rate of emission
\begin{equation}
  \Rem = \sum_{j=1}^{\ncl} \sum_{g} \Remg
\end{equation}
where
\begin{equation}
  \Remg = \int_{\nu_{g-1/2}^\nu_{g+1/2}} B(\nu,t) 
\end{equation}
\item Sample the time of the next event from $i
\item Sample random number $\xi\sim\text{U}(0,1)$
\item If $\xi > \frac{\Rint}}{R_{t}}/  if interaction (including 'virtual' interactions) or emission event
from 

\item If emission:
 \begin{enumerate}

  \end{enumerate}
\end{enumerate}


\end{enumerate}

\section{Other stuff I need to write up/think about}

\begin{itemize}
  \item What happens when a new photon is emitted
  \item What happens if we have cooling (i.e., a cross section increase for an
active photon)
  \item How to write probability event for time greater than leakage time.
  \item How to write 
\end{itemize}


%%%%%%%%%%%%%%%%%%%%%%%%%%%%%%%%%%%%%%%%%%%%%%%%%%%%%%%%%%%%%%%%%%%%%%%%%%%%%%%%
\bibliographystyle{ans}
\bibliography{references}
\end{document}

